\documentclass[12pt, letterpaper]{article}
\usepackage{color}   %May be necessary if you want to color links
\usepackage{hyperref}
\hypersetup{
    colorlinks=true, %set true if you want colored links
    linktoc=all,     %set to all if you want both sections and subsections linked
    linkcolor=black,  %choose some color if you want links to stand out
}
\usepackage[utf8x]{inputenc}
\usepackage[english]{babel}
\usepackage{amsmath, amsthm, amssymb, amsfonts}
\usepackage{graphicx}
\graphicspath{ {./images/} }
\newcommand{\img}[3] {
 	\begin{figure}[h]
  		\caption{#1}
		\centering
		\includegraphics[scale=#2]{#3}\\
	\end{figure}
}
\title{La Bibbia di Sistemi operativi}
\author{Mario Petruccelli \cr Università degli studi di Milano}
\date{A.A. 2018/2019}

\addto\captionsenglish{
	\renewcommand{\contentsname}%
    	{Sommario}
}

\begin{document}

\begin{titlepage}
\maketitle
\newpage
\tableofcontents
\newpage

\section{Processo} 
	\subsubsection{Che cos’è un processo? A che cosa serve?}
		Inserire risposta qui \dots
	\subsubsection{Perché è stato introdotto il concetto di processo?}
	\subsubsection{Potrebbero esistere dei sistemi operativi senza processi?}
	\subsubsection{Perché non serve il processo in un sistema mono programmato?}
	\subsubsection{MS-DOS aveva bisogno del processo o no? Perché?}
	\subsubsection{Associata all’astrazione del processo, cosa c’è? Come lorappresentiamo  all’interno  del  processo?}
	\subsubsection{Cos’è  un  processo?}
	\subsubsection{Quando  si  è  resa  necessaria  l’astrazione  dei  processi?}
	\subsubsection{Come vengono gestiti i processi  in  un  sistema  multiprogrammato?} 


\end{titlepage}


\end{document}